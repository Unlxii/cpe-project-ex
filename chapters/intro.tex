\chapter{\ifenglish Introduction\else บทนำ\fi}

\section{\ifenglish Project rationale\else ที่มาของโครงงาน\fi}


ในยุคที่การพัฒนาซอฟต์แวร์มีความรวดเร็วและซับซ้อนมากขึ้น\linebreak การนำเทคโนโลยี DevOps มาใช้ในองค์กรต่าง ๆ เป็นสิ่งที่ช่วยเพิ่มประสิทธิภาพในการพัฒนาและปรับใช้แอปพลิเคชัน\linebreak อย่างไรก็ตาม ความรวดเร็วในการพัฒนามักจะมาพร้อมกับความเสี่ยงด้านความปลอดภัยที่อาจถูกละเลย\linebreak ตามรายงานของ OWASP Top 10~\cite{owasp2021} พบว่าช่องโหว่ด้านความปลอดภัยในแอปพลิเคชันเว็บยังคงเป็นปัญหาสำคัญที่ส่งผลกระทบต่อองค์กรต่าง ๆ ทั่วโลก\linebreak นอกจากนี้ การใช้เทคโนโลยีคอนเทนเนอร์อย่าง Docker และ Kubernetes ที่กำลังเป็นที่นิยมในปัจจุบันก็ยังมีความเสี่ยงด้านความปลอดภัยที่ต้องคำนึงถึง\linebreak ตามที่ NIST SP 800-190~\cite{nist800-190} ได้ให้คำแนะนำเกี่ยวกับการรักษาความปลอดภัยสำหรับแอปพลิเคชันที่ทำงานบนคอนเทนเนอร์

VisOperation เป็นแพลตฟอร์มที่พัฒนาขึ้นเพื่อสนับสนุนการทำงานด้าน DevOps และการจัดการโครงสร้างพื้นฐาน อย่างไรก็ตาม แพลตฟอร์มยังขาดความสามารถในการตรวจสอบความปลอดภัยของซอร์สโค้ดและคอนเทนเนอร์อย่างเป็นระบบ ซึ่งเป็นส่วนสำคัญของการนำ DevSecOps มาใช้ในองค์กร ตามมาตรฐาน ISO/IEC 27001:2022~\cite{iso27001} ซึ่งเน้นการจัดการความปลอดภัยของข้อมูลและระบบสารสนเทศ การตรวจสอบช่องโหว่และการจัดการความเสี่ยงเป็นส่วนสำคัญที่องค์กรต้องดำเนินการอย่างต่อเนื่อง

จากปัญหาและความต้องการดังกล่าว ทำให้เกิดแนวคิดในการพัฒนาระบบ VisScan ซึ่งเป็นศูนย์ตรวจสอบความปลอดภัยโค้ดและคอนเทนเนอร์ที่ออกแบบมาเพื่อเชื่อมต่อกับแพลตฟอร์ม VisOperation โดยระบบนี้จะทำหน้าที่ตรวจสอบช่องโหว่ในซอร์สโค้ด การรั่วไหลของข้อมูลลับ (Secret Leakage) และความปลอดภัยของอิมเมจคอนเทนเนอร์ ด้วยการใช้เครื่องมือมาตรฐานอุตสาหกรรมที่มีความน่าเชื่อถือ เช่น Trivy~\cite{trivy}, Gitleaks~\cite{gitleaks} และ Semgrep~\cite{semgrep} ระบบนี้จะช่วยให้องค์กรสามารถตรวจจับและแก้ไขปัญหาด้านความปลอดภัยได้อย่างรวดเร็วในทุกขั้นตอนของวงจรการพัฒนาซอฟต์แวร์

\section{\ifenglish Objectives\else วัตถุประสงค์ของโครงงาน\fi}
\begin{enumerate}
    \item เพื่อพัฒนาระบบตรวจสอบความปลอดภัยของซอร์สโค้ดและอิมเมจคอนเทนเนอร์ที่สามารถทำงานร่วมกับแพลตฟอร์ม VisOperation ได้อย่างมีประสิทธิภาพ
    \item เพื่อออกแบบและพัฒนาระบบที่สามารถตรวจจับช่องโหว่ด้านความปลอดภัย (Vulnerabilities) การรั่วไหลของข้อมูลลับ (Secret Leakage) และความผิดพลาดในการตั้งค่า (Misconfiguration) ได้อย่างครอบคลุม
    \item เพื่อพัฒนาระบบที่รองรับการทำงานกับเครื่องมือตรวจสอบความปลอดภัยหลากหลายประเภท ได้แก่ Trivy, Gitleaks และ Semgrep โดยสามารถบูรณาการและนำเสนอผลการตรวจสอบในรูปแบบที่เป็นมาตรฐาน
    \item เพื่อพัฒนาระบบจัดการและนำเสนอผลการสแกน (Scan Result Management) ที่ช่วยให้ทีมพัฒนาและทีมปฏิบัติการสามารถติดตามและแก้ไขปัญหาด้านความปลอดภัยได้อย่างมีประสิทธิภาพ
    \item เพื่อพัฒนาระบบแจ้งเตือน (Notification System) และเชื่อมต่อเข้ากับ CI/CD Pipeline เพื่อให้สามารถตรวจสอบความปลอดภัยได้อย่างอัตโนมัติในทุกขั้นตอนของการพัฒนา
    \item เพื่อศึกษาและนำมาตรฐานสากลด้านความปลอดภัย เช่น ISO/IEC 27001, NIST SP 800-190 และ OWASP มาประยุกต์ใช้ในการออกแบบและพัฒนาระบบ
\end{enumerate}

\section{\ifenglish Project scope\else ขอบเขตของโครงงาน\fi}

\subsection{\ifenglish Hardware scope\else ขอบเขตด้านฮาร์ดแวร์\fi}

\begin{enumerate}
    \item เซิร์ฟเวอร์สำหรับติดตั้งและรันระบบ\hspace{0pt} VisScan\hspace{0pt} ที่มีทรัพยากรเพียงพอสำหรับการทำงานของ\hspace{0pt} Docker\hspace{0pt} Engine และ\hspace{0pt} Kubernetes
    \item เซิร์ฟเวอร์ฐานข้อมูล PostgreSQL สำหรับจัดเก็บข้อมูลผลการสแกนและข้อมูลการตั้งค่าระบบ
    \item เครือข่ายที่เชื่อมต่อกับแพลตฟอร์ม VisOperation และ Git Repository ต่าง ๆ
    \item เครื่องคอมพิวเตอร์สำหรับพัฒนาและทดสอบระบบ
\end{enumerate}

\subsection{\ifenglish Software scope\else ขอบเขตด้านซอฟต์แวร์\fi}

\begin{enumerate}
    \item ระบบ Frontend: พัฒนาด้วย Next.js, React และ Tailwind CSS สำหรับแสดงผลการสแกนและจัดการการตั้งค่า โดยเชื่อมต่อกับ VisOperation Platform
    \item ระบบ Backend: พัฒนาด้วย Node.js และ Prisma ORM เพื่อจัดการ API, ประมวลผลการสแกน และจัดการฐานข้อมูล
    \item ระบบฐานข้อมูล: ใช้ PostgreSQL สำหรับจัดเก็บข้อมูลโปรเจกต์ ผลการสแกน และประวัติการตรวจสอบ
    \item Security Scanning Engines:
    \begin{itemize}
        \item Trivy: ตรวจสอบช่องโหว่ในคอนเทนเนอร์อิมเมจ แพ็กเกจของระบบปฏิบัติการ และไลบรารี่ต่าง ๆ
        \item Gitleaks: ตรวจจับข้อมูลลับที่รั่วไหล เช่น API Keys, Passwords, Tokens ในซอร์สโค้ด
        \item Semgrep: ทำ Static Application Security Testing (SAST) เพื่อค้นหาช่องโหว่ในโค้ด
    \end{itemize}
    \item ระบบ Container Orchestration: Docker และ Kubernetes สำหรับการจัดการและ Deploy แอปพลิเคชัน
    \item ระบบ Notification: เชื่อมต่อกับ Email, Slack หรือระบบแจ้งเตือนอื่น ๆ
    \item CI/CD Integration: เชื่อมต่อเข้ากับ Pipeline เพื่อให้สามารถทำการสแกนอัตโนมัติได้
\end{enumerate}

\section{\ifenglish Expected outcomes\else ประโยชน์ที่ได้รับ\fi}

\begin{enumerate}
    \item องค์กรได้ระบบตรวจสอบความปลอดภัยที่สามารถทำงานร่วมกับแพลตฟอร์ม\hspace{0pt} VisOperation\hspace{0pt} ได้อย่างสมบูรณ์ ช่วยเสริมศักยภาพด้าน DevSecOps
    \item ทีมพัฒนาสามารถตรวจจับช่องโหว่ด้านความปลอดภัยได้ตั้งแต่ในขั้นตอนการพัฒนา (Shift-Left Security) ลดต้นทุนในการแก้ไขปัญหาในภายหลัง
    \item ลดความเสี่ยงจากการรั่วไหลของข้อมูลลับ เช่น API Keys, Database Passwords ที่อาจถูกฝังไว้ในซอร์สโค้ดโดยไม่ตั้งใจ
    \item เพิ่มความโปร่งใสในการตรวจสอบความปลอดภัยของคอนเทนเนอร์และแอปพลิเคชัน ทำให้องค์กรสามารถปฏิบัติตามมาตรฐานความปลอดภัย เช่น ISO/IEC 27001 ได้ดีขึ้น
    \item ระบบสามารถบูรณาการเข้ากับ CI/CD Pipeline ทำให้การตรวจสอบความปลอดภัยเป็นส่วนหนึ่งของกระบวนการพัฒนาอัตโนมัติ
    \item ทีมปฏิบัติการได้รับข้อมูลที่ครอบคลุมเกี่ยวกับความปลอดภัยของระบบ สามารถจัดลำดับความสำคัญและแก้ไขปัญหาได้อย่างมีประสิทธิภาพ
    \item เป็นต้นแบบสำหรับการพัฒนาระบบความปลอดภัยแบบบูรณาการสำหรับแพลตฟอร์ม DevOps อื่น ๆ
\end{enumerate}

\section{\ifenglish Technology and tools\else เทคโนโลยีและเครื่องมือที่ใช้\fi}

\subsection{\ifenglish Hardware technology\else เทคโนโลยีด้านฮาร์ดแวร์\fi}

\begin{enumerate}
    \item เซิร์ฟเวอร์สำหรับรันแอปพลิเคชัน: CPU 4 cores ขึ้นไป, RAM 8 GB ขึ้นไป, Storage 100 GB ขึ้นไป
    \item เซิร์ฟเวอร์ฐานข้อมูล PostgreSQL: CPU 2 cores ขึ้นไป, RAM 4 GB ขึ้นไป, Storage 50 GB ขึ้นไป
    \item Network Infrastructure: รองรับการสื่อสารแบบ HTTPS และมีระบบ Firewall
\end{enumerate}

\subsection{\ifenglish Software technology\else เทคโนโลยีด้านซอฟต์แวร์\fi}

\begin{enumerate}
    \item \textbf{Frontend Development}
    \begin{itemize}
        \item Next.js 14+: React Framework สำหรับพัฒนา Web Application
        \item React 18+: JavaScript Library สำหรับสร้าง User Interface
        \item Tailwind CSS 3+: Utility-first CSS Framework
        \item TypeScript: สำหรับ Type Safety ในการพัฒนา
    \end{itemize}
    
    \item \textbf{Backend Development}
    \begin{itemize}
        \item Node.js 20+: JavaScript Runtime Environment
        \item Prisma ORM: Object-Relational Mapping สำหรับจัดการฐานข้อมูล
        \item Express.js / Fastify: Web Framework สำหรับ API Development
    \end{itemize}
    
    \item \textbf{Database}
    \begin{itemize}
        \item PostgreSQL 15+: Relational Database Management System
    \end{itemize}
    
    \item \textbf{Security Scanning Tools}
    \begin{itemize}
        \item Trivy: Container Image และ Filesystem Vulnerability Scanner
        \item Gitleaks: Secret Detection Tool
        \item Semgrep: Static Analysis Engine สำหรับ SAST
    \end{itemize}
    
    \item \textbf{DevOps \& Infrastructure}
    \begin{itemize}
        \item Docker: Container Platform
        \item Kubernetes: Container Orchestration
        \item Git: Version Control System
        \item CI/CD Tools: Jenkins, GitLab CI, GitHub Actions
    \end{itemize}
\end{enumerate}


\section{\ifenglish Project plan\else แผนการดำเนินงาน\fi}

\begin{plan}{9}{2025}{3}{2026}
    \planitem{9}{2025}{9}{2025}{ศึกษาค้นคว้าข้อมูลเกี่ยวกับโปรเจกต์}
    \planitem{10}{2025}{10}{2025}{ปรึกษาอาจารย์และวางแผน เตรียม environments}
    \planitem{11}{2025}{11}{2025}{พัฒนา prototype รุ่นแรก และทดสอบการใช้งาน}
    \planitem{12}{2025}{12}{2025}{รับ requirement เพิ่มเติม และ ปรับปรุงการทำงาน}
    \planitem{1}{2026}{1}{2026}{นำเสนอผลงานและรับ feedback}
    \planitem{1}{2026}{1}{2026}{วางแผนปรับปรุงและพัฒนา}
    \planitem{2}{2026}{2}{2026}{เชื่อมต่อกับ VisOperation Platform}
    \planitem{3}{2026}{3}{2026}{}
\end{plan}

\section{\ifenglish Roles and responsibilities\else บทบาทและความรับผิดชอบ\fi}

การพัฒนาโครงงาน VisScan ต้องอาศัยความรู้และทักษะในหลายด้าน ได้แก่

\begin{enumerate}
    \item \textbf{ด้านการออกแบบระบบ}: นักศึกษาทั้งสองคนร่วมกันออกแบบสถาปัตยกรรมระบบ วางแผนการทำงานของ Microservices และออกแบบฐานข้อมูล โดยใช้ความรู้จากวิชา Software Engineering และ Database Systems
    
    \item \textbf{ด้านการพัฒนา Frontend}: รับผิดชอบพัฒนา User Interface ด้วย Next.js และ React โดยใช้ความรู้จากวิชา Web Development และ User Interface Design
    
    \item \textbf{ด้านการพัฒนา Backend}: รับผิดชอบพัฒนา API และระบบประมวลผล โดยใช้ความรู้จากวิชา Web Programming, API Design และ System Programming
    
    \item \textbf{ด้านความปลอดภัย}: ทั้งสองคนร่วมกันศึกษาและบูรณาการเครื่องมือด้านความปลอดภัย โดยใช้ความรู้จากวิชา Computer Security และ Network Security
    
    \item \textbf{ด้าน DevOps}: รับผิดชอบการติดตั้ง การตั้งค่า Docker, Kubernetes และ CI/CD Pipeline โดยใช้ความรู้จากวิชา Operating Systems และ Cloud Computing
    
    \item \textbf{ด้านการทดสอบ}: ทั้งสองคนร่วมกันทดสอบระบบ Unit Testing, Integration Testing และ Security Testing โดยใช้ความรู้จากวิชา Software Testing
\end{enumerate}

การแบ่งหน้าที่ความรับผิดชอบจะมีความยืดหยุ่น โดยทั้งสองคนจะช่วยเหลือซึ่งกันและกันในทุกส่วนของโครงงาน และทำงานร่วมกันในการแก้ไขปัญหาที่เกิดขึ้น

\section{\ifenglish%
Impacts of this project on society, health, safety, legal, and cultural issues
\else%
ผลกระทบด้านสังคม สุขภาพ ความปลอดภัย กฎหมาย และวัฒนธรรม
\fi}

\subsection{ผลกระทบด้านสังคม}
ระบบ VisScan ช่วยเพิ่มความตระหนักรู้ด้านความปลอดภัยในการพัฒนาซอฟต์แวร์ให้กับองค์กรและนักพัฒนา ส่งเสริมวัฒนธรรมการพัฒนาที่คำนึงถึงความปลอดภัยตั้งแต่เริ่มต้น (Security-First Culture) ซึ่งจะช่วยลดความเสี่ยงที่จะส่งผลกระทบต่อผู้ใช้งานและสังคมโดยรวม

\subsection{ผลกระทบด้านสุขภาพและความปลอดภัย}
การป้องกันช่องโหว่ด้านความปลอดภัยช่วยลดความเสี่ยงต่อการถูกโจมตีทางไซเบอร์ ซึ่งอาจส่งผลกระทบต่อระบบสุขภาพ โรงพยาบาล หรือระบบโครงสร้างพื้นฐานที่สำคัญ การรักษาความปลอดภัยของข้อมูลผู้ป่วยและข้อมูลสุขภาพเป็นสิ่งสำคัญที่ระบบนี้สามารถช่วยปกป้องได้

\subsection{ผลกระทบด้านกฎหมาย}
ระบบช่วยให้องค์กรสามารถปฏิบัติตามกฎหมายและข้อบังคับด้านความปลอดภัยข้อมูล เช่น พ.ร.บ. คุ้มครองข้อมูลส่วนบุคคล (PDPA) และมาตรฐานสากล เช่น ISO/IEC 27001 ลดความเสี่ยงทางกฎหมายและค่าปรับที่อาจเกิดจากการละเมิดความปลอดภัย

\subsection{ผลกระทบด้านวัฒนธรรมองค์กร}
การนำระบบ VisScan มาใช้ช่วยสร้างวัฒนธรรมความรับผิดชอบร่วมกันด้านความปลอดภัย (Shared Security Responsibility) ระหว่างทีมพัฒนาและทีมปฏิบัติการ ส่งเสริมการทำงานแบบ DevSecOps ที่บูรณาการความปลอดภัยเข้ากับกระบวนการพัฒนา

\subsection{การประยุกต์ใช้งาน}
ระบบ VisScan สามารถนำไปประยุกต์ใช้ได้ในหลากหลายอุตสาหกรรม เช่น การเงินและธนาคาร, สุขภาพและการแพทย์, รัฐบาลและหน่วยงานภาครัฐ, อีคอมเมิร์ซ และองค์กรที่ให้บริการคลาวด์ โดยช่วยเพิ่มความมั่นคงปลอดภัยของระบบและข้อมูล ลดความเสี่ยงจากภัยคุกคามทางไซเบอร์ และสร้างความเชื่อมั่นให้กับผู้ใช้บริการ
