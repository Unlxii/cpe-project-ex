\chapter{\ifproject%
\ifenglish Project Structure and Methodology\else โครงสร้างและขั้นตอนการทำงาน\fi
\else%
\ifenglish Project Structure\else โครงสร้างของโครงงาน\fi
\fi
}

ในบทนี้จะอธิบายถึงสถาปัตยกรรมของระบบ VisScan การออกแบบโครงสร้าง และขั้นตอนการทำงานของระบบแต่ละส่วน รวมถึงวิธีการบูรณาการเข้ากับแพลตฟอร์ม VisOperation

\section{สถาปัตยกรรมระบบโดยรวม}

ระบบ VisScan ออกแบบตามแนวคิด Microservices Architecture เพื่อให้มีความยืดหยุ่น ขยายได้ง่าย และสามารถบำรุงรักษาได้สะดวก โดยประกอบด้วยส่วนประกอบหลักดังนี้:

\subsection{ส่วนประกอบหลักของระบบ}

\begin{enumerate}
    \item \textbf{Frontend Layer (Web Application)}
    \begin{itemize}
        \item พัฒนาด้วย Next.js และ React
        \item ใช้ Tailwind CSS สำหรับ UI/UX Design
        \item บูรณาการเข้ากับ VisOperation Platform ผ่าน SSO และ API Gateway
        \item แสดงผลการสแกน Dashboard, Report, และ Configuration Interface
    \end{itemize}
    
    \item \textbf{API Gateway}
    \begin{itemize}
        \item ทำหน้าที่เป็นจุดเชื่อมต่อหลักระหว่าง Frontend และ Backend Services
        \item จัดการ Authentication และ Authorization ร่วมกับ VisOperation
        \item Route Request ไปยัง Backend Service ที่เหมาะสม
        \item Rate Limiting และ Request Validation
    \end{itemize}
    
    \item \textbf{Backend Services}
    \begin{itemize}
        \item \textbf{Scan Orchestration Service}: จัดการ Workflow การสแกน, Queue Management, และ Job Scheduling
        \item \textbf{Scanner Manager Service}: ควบคุมและเรียกใช้เครื่องมือสแกน (Trivy, Gitleaks, Semgrep)
        \item \textbf{Result Processing Service}: ประมวลผล Normalize และจัดเก็บผลการสแกน
        \item \textbf{Notification Service}: จัดการการแจ้งเตือนผ่าน Email, Slack, Webhook
        \item \textbf{Project Management Service}: จัดการข้อมูลโปรเจกต์ และการตั้งค่า
    \end{itemize}
    
    \item \textbf{Database Layer}
    \begin{itemize}
        \item PostgreSQL สำหรับเก็บข้อมูล Structured Data
        \item Schema ออกแบบด้วย Prisma ORM
        \item รองรับ Transaction และ ACID Properties
    \end{itemize}
    
    \item \textbf{Storage Layer}
    \begin{itemize}
        \item File Storage สำหรับเก็บ Scan Reports และ Source Code ชั่วคราว
        \item Container Registry Integration สำหรับดึง Container Image
    \end{itemize}
    
    \item \textbf{Security Scanning Engines}
    \begin{itemize}
        \item Trivy Container: รันใน Isolated Container
        \item Gitleaks Container: รันใน Isolated Container
        \item Semgrep Container: รันใน Isolated Container
    \end{itemize}
\end{enumerate}

\subsection{การบูรณาการกับ VisOperation Platform}

VisScan ออกแบบมาเพื่อทำงานร่วมกับ VisOperation อย่างไร้รอยต่อ:

\begin{itemize}
    \item \textbf{Single Sign-On (SSO)}: ใช้ระบบ Authentication เดียวกันกับ VisOperation ผู้ใช้ไม่ต้อง Login ซ้ำ
    \item \textbf{Unified Dashboard}: VisScan Dashboard แสดงผลภายใน VisOperation Platform
    \item \textbf{API Integration}: VisScan ให้บริการ REST API สำหรับให้ VisOperation เรียกใช้
    \item \textbf{Shared Database}: บางข้อมูลเช่น Project, User, Permission ใช้ร่วมกับ VisOperation
    \item \textbf{Webhook Integration}: VisScan รับ Event จาก VisOperation เช่น New Deployment, Git Push
\end{itemize}

\section{Data Flow และ Workflow}

\subsection{Workflow การสแกนซอร์สโค้ด}

ขั้นตอนการทำงานเมื่อผู้ใช้ต้องการสแกนซอร์สโค้ด:

\begin{enumerate}
    \item \textbf{Initiation}: ผู้ใช้เริ่มต้นการสแกนผ่าน Web UI หรือ API โดยระบุ Git Repository URL หรืออัปโหลด Source Code
    
    \item \textbf{Authentication \& Authorization}: ระบบตรวจสอบสิทธิ์ผู้ใช้และสิทธิ์เข้าถึง Repository
    
    \item \textbf{Code Retrieval}: ระบบ Clone หรือ Download ซอร์สโค้ดมาเก็บไว้ใน Temporary Storage
    
    \item \textbf{Scan Job Creation}: Scan Orchestration Service สร้าง Scan Job และเพิ่มเข้า Queue
    
    \item \textbf{Scanner Execution}: Scanner Manager Service เรียกใช้เครื่องมือสแกนตามที่กำหนด:
    \begin{itemize}
        \item Gitleaks สำหรับตรวจจับ Secret Leakage
        \item Semgrep สำหรับ SAST
        \item Trivy สำหรับ Dependency Scanning (SCA)
    \end{itemize}
    
    \item \textbf{Result Collection}: แต่ละ Scanner ส่งผลลัพธ์กลับมาในรูปแบบ JSON
    
    \item \textbf{Result Normalization}: Result Processing Service แปลงผลลัพธ์จากแต่ละเครื่องมือให้เป็นรูปแบบมาตรฐาน
    
    \item \textbf{Severity Classification}: จัดลำดับความรุนแรงของช่องโหว่เป็น Critical, High, Medium, Low
    
    \item \textbf{Data Storage}: บันทึกผลลัพธ์ลงฐานข้อมูล PostgreSQL
    
    \item \textbf{Notification}: หากพบปัญหาที่มีความรุนแรงสูง ระบบจะส่งการแจ้งเตือน
    
    \item \textbf{Cleanup}: ลบ Source Code ชั่วคราวออกจากระบบ
    
    \item \textbf{Report Generation}: สร้างรายงานสรุปผลการสแกน
\end{enumerate}

\subsection{Workflow การสแกน Container Image}

ขั้นตอนการทำงานเมื่อผู้ใช้ต้องการสแกน Container Image:

\begin{enumerate}
    \item \textbf{Image Specification}: ผู้ใช้ระบุ Container Image ที่ต้องการสแกน (Image Name:Tag)
    
    \item \textbf{Registry Authentication}: หาก Image อยู่ใน Private Registry ระบบจะใช้ Credentials ที่กำหนดไว้
    
    \item \textbf{Image Pulling}: ดึง Container Image จาก Registry มาเก็บไว้ชั่วคราว (หรือใช้ Remote Scan)
    
    \item \textbf{Trivy Scanning}: เรียกใช้ Trivy ในการสแกนหา:
    \begin{itemize}
        \item OS Package Vulnerabilities
        \item Application Library Vulnerabilities
        \item Misconfiguration in Dockerfile
        \item Sensitive Information
    \end{itemize}
    
    \item \textbf{SBOM Generation}: Trivy สร้าง Software Bill of Materials (SBOM) ระบุ Component ทั้งหมดใน Image
    
    \item \textbf{Vulnerability Matching}: เปรียบเทียบ Component กับฐานข้อมูลช่องโหว่
    
    \item \textbf{Result Processing}: ประมวลผลและจัดกลุ่มช่องโหว่ตามความรุนแรง
    
    \item \textbf{Storage \& Notification}: เก็บผลลัพธ์และแจ้งเตือนเช่นเดียวกับ Source Code Scanning
\end{enumerate}

\subsection{CI/CD Pipeline Integration Workflow}

การทำงานเมื่อบูรณาการเข้ากับ CI/CD Pipeline:

\begin{enumerate}
    \item \textbf{Pipeline Trigger}: CI/CD Pipeline เรียกใช้ VisScan API เมื่อมีการ Build หรือ Deploy
    
    \item \textbf{Pre-deployment Scan}: VisScan สแกน Image ก่อน Deploy จริง
    
    \item \textbf{Policy Enforcement}: ตรวจสอบว่าผลการสแกนผ่านเกณฑ์ที่กำหนดหรือไม่:
    \begin{itemize}
        \item ไม่มีช่องโหว่ระดับ Critical
        \item จำนวนช่องโหว่ระดับ High ไม่เกินที่กำหนด
        \item ไม่มี Secret Leakage
    \end{itemize}
    
    \item \textbf{Pipeline Decision}: ส่งผลลัพธ์กลับไปยัง Pipeline:
    \begin{itemize}
        \item Pass: อนุญาตให้ Deploy
        \item Fail: หยุด Pipeline และแจ้งเตือนทีมพัฒนา
    \end{itemize}
\end{enumerate}

\section{Database Schema Design}

โครงสร้างฐานข้อมูลออกแบบด้วย Prisma ORM ประกอบด้วย Entity หลัก:

\subsection{Entity หลัก}

\begin{enumerate}
    \item \textbf{Project}
    \begin{itemize}
        \item id, name, description
        \item repository\_url, branch
        \item created\_at, updated\_at
        \item owner\_id (Foreign Key)
    \end{itemize}
    
    \item \textbf{Scan}
    \begin{itemize}
        \item id, project\_id (Foreign Key)
        \item scan\_type (source\_code, container\_image)
        \item status (pending, running, completed, failed)
        \item started\_at, completed\_at
        \item triggered\_by (user, ci\_cd, scheduled)
    \end{itemize}
    
    \item \textbf{Vulnerability}
    \begin{itemize}
        \item id, scan\_id (Foreign Key)
        \item scanner\_name (trivy, gitleaks, semgrep)
        \item title, description
        \item severity (critical, high, medium, low, info)
        \item cve\_id (ถ้ามี)
        \item file\_path, line\_number
        \item remediation\_advice
    \end{itemize}
    
    \item \textbf{Secret}
    \begin{itemize}
        \item id, scan\_id (Foreign Key)
        \item secret\_type (api\_key, password, token, etc.)
        \item file\_path, line\_number
        \item match\_string (masked)
        \item confidence\_level
    \end{itemize}
    
    \item \textbf{ContainerImage}
    \begin{itemize}
        \item id, project\_id (Foreign Key)
        \item image\_name, image\_tag
        \item registry\_url
        \item digest (SHA256)
        \item size\_bytes
        \item created\_at
    \end{itemize}
    
    \item \textbf{Notification}
    \begin{itemize}
        \item id, scan\_id (Foreign Key)
        \item notification\_type (email, slack, webhook)
        \item recipient, status
        \item sent\_at
    \end{itemize}
    
    \item \textbf{Policy}
    \begin{itemize}
        \item id, project\_id (Foreign Key)
        \item max\_critical, max\_high
        \item block\_on\_secret\_leak (boolean)
        \item notification\_settings (JSON)
    \end{itemize}
\end{enumerate}

\subsection{ความสัมพันธ์ระหว่าง Entity}

\begin{itemize}
    \item Project 1:N Scan (หนึ่ง Project มีหลาย Scan)
    \item Scan 1:N Vulnerability (หนึ่ง Scan พบหลาย Vulnerability)
    \item Scan 1:N Secret (หนึ่ง Scan พบหลาย Secret)
    \item Project 1:N ContainerImage (หนึ่ง Project มีหลาย Image)
    \item Project 1:1 Policy (หนึ่ง Project มีหนึ่ง Policy)
\end{itemize}

\section{API Design}

VisScan ให้บริการ RESTful API สำหรับการเชื่อมต่อกับระบบอื่น

\subsection{Core API Endpoints}

\begin{enumerate}
    \item \textbf{Scan Management}
    \begin{itemize}
        \item POST /api/v1/scans/source-code - เริ่มการสแกนซอร์สโค้ด
        \item POST /api/v1/scans/container-image - เริ่มการสแกน Container Image
        \item GET /api/v1/scans/\{scanId\} - ดูข้อมูล Scan
        \item GET /api/v1/scans/\{scanId\}/status - ตรวจสอบสถานะ Scan
        \item GET /api/v1/scans/\{scanId\}/results - ดูผลการสแกน
    \end{itemize}
    
    \item \textbf{Project Management}
    \begin{itemize}
        \item GET /api/v1/projects - ดูรายการ Project
        \item POST /api/v1/projects - สร้าง Project ใหม่
        \item GET /api/v1/projects/\{projectId\} - ดูข้อมูล Project
        \item PUT /api/v1/projects/\{projectId\} - แก้ไข Project
        \item DELETE /api/v1/projects/\{projectId\} - ลบ Project
    \end{itemize}
    
    \item \textbf{Vulnerability Management}
    \begin{itemize}
        \item GET /api/v1/vulnerabilities - ดูรายการช่องโหว่ทั้งหมด
        \item GET /api/v1/vulnerabilities/\{vulnId\} - ดูรายละเอียดช่องโหว่
        \item PUT /api/v1/vulnerabilities/\{vulnId\}/status - อัปเดตสถานะ (acknowledged, fixed, false\_positive)
    \end{itemize}
    
    \item \textbf{Policy Management}
    \begin{itemize}
        \item GET /api/v1/projects/\{projectId\}/policy - ดู Policy
        \item PUT /api/v1/projects/\{projectId\}/policy - อัปเดต Policy
    \end{itemize}
    
    \item \textbf{Webhook Integration}
    \begin{itemize}
        \item POST /api/v1/webhooks/github - รับ Event จาก GitHub
        \item POST /api/v1/webhooks/gitlab - รับ Event จาก GitLab
    \end{itemize}
\end{enumerate}

\section{Security Design}

\subsection{Authentication และ Authorization}

\begin{itemize}
    \item ใช้ JWT (JSON Web Token) สำหรับ Authentication
    \item บูรณาการกับ VisOperation SSO
    \item Role-Based Access Control (RBAC): Admin, Developer, Viewer
    \item API Key สำหรับ CI/CD Integration
\end{itemize}

\subsection{Data Protection}

\begin{itemize}
    \item เข้ารหัสข้อมูลลับใน Database (Credentials, API Keys) ด้วย AES-256
    \item ใช้ HTTPS สำหรับการสื่อสารทั้งหมด
    \item Automatic cleanup ของ Source Code และ Sensitive Data หลังสแกนเสร็จ
    \item Access Logging และ Audit Trail
\end{itemize}

\subsection{Isolation และ Resource Management}

\begin{itemize}
    \item แต่ละ Scan Job รันใน Isolated Container
    \item Resource Limits (CPU, Memory, Storage) สำหรับแต่ละ Scanner
    \item Timeout Mechanism ป้องกัน Long-running Jobs
    \item Network Isolation ระหว่าง Scanner Container
\end{itemize}

\section{Scalability และ Performance}

\subsection{Horizontal Scaling}

\begin{itemize}
    \item Backend Services สามารถ Scale Out ได้อิสระ
    \item Job Queue (Redis/RabbitMQ) สำหรับ Distribute Scan Jobs
    \item Multiple Scanner Instances ทำงานพร้อมกัน
\end{itemize}

\subsection{Caching Strategy}

\begin{itemize}
    \item Cache ผลการสแกนสำหรับ Image ที่มี Digest เดียวกัน
    \item Cache ฐานข้อมูลช่องโหว่ของ Trivy (อัปเดตทุก 6 ชั่วโมง)
    \item Redis Cache สำหรับ API Response ที่ Query บ่อย
\end{itemize}

\subsection{Performance Optimization}

\begin{itemize}
    \item Parallel Scanning: รัน Trivy, Gitleaks, Semgrep พร้อมกัน
    \item Incremental Scanning: สแกนเฉพาะไฟล์ที่เปลี่ยนแปลง (สำหรับ Source Code)
    \item Database Indexing สำหรับ Query ที่ใช้บ่อย
    \item Async Processing สำหรับงานที่ไม่ต้องการผลลัพธ์ทันที
\end{itemize}

\section{สรุป}

การออกแบบสถาปัตยกรรมของ VisScan คำนึงถึงความยืดหยุ่น ความปลอดภัย และประสิทธิภาพในการทำงาน การใช้ Microservices Architecture ทำให้สามารถขยายและบำรุงรักษาแต่ละส่วนได้อิสระ การบูรณาการกับ VisOperation Platform ทำให้ผู้ใช้สามารถเข้าถึงฟีเจอร์ความปลอดภัยได้อย่างสะดวก และการออกแบบ Workflow ที่มีประสิทธิภาพช่วยให้สามารถตรวจจับช่องโหว่ได้อย่างรวดเร็วและแม่นยำ
...existing code...
% ลบเนื้อหาตัวอย่าง The Black Kitten, The Reproach และข้อความภาษาอังกฤษที่ไม่เกี่ยวข้อง
