\chapter{\ifenglish Background Knowledge and Theory\else ทฤษฎีที่เกี่ยวข้อง\fi}

ในบทนี้จะกล่าวถึงทฤษฎี แนวคิด และเทคโนโลยีที่เกี่ยวข้องกับการพัฒนาระบบ VisScan รวมถึงมาตรฐานสากลด้านความปลอดภัยที่นำมาประยุกต์ใช้ในโครงงาน เพื่อให้ผู้อ่านมีความเข้าใจพื้นฐานที่จำเป็นก่อนศึกษารายละเอียดของระบบในบทถัดไป

\section{DevSecOps และหลักการ Shift-Left Security}

DevSecOps เป็นแนวทางในการพัฒนาซอฟต์แวร์ที่บูรณาการความปลอดภัย (Security) เข้ากับกระบวนการ Development และ Operations อย่างเป็นระบบ~\cite{devsecops2019} โดยมีเป้าหมายหลักคือการทำให้ความปลอดภัยเป็นความรับผิดชอบร่วมกันของทุกคนในทีม ไม่ใช่แค่ทีมความปลอดภัยเท่านั้น

\subsection{หลักการสำคัญของ DevSecOps}

\begin{enumerate}
    \item \textbf{Automation}: การตรวจสอบความปลอดภัยต้องทำงานอัตโนมัติและเป็นส่วนหนึ่งของ CI/CD Pipeline เพื่อไม่ให้เป็นอุปสรรคต่อความเร็วในการพัฒนา
    
    \item \textbf{Shift-Left Security}: การเลื่อนการตรวจสอบความปลอดภัยมาทำในช่วงต้นของวงจรการพัฒนา (Development Phase) แทนที่จะรอจนถึงช่วงท้าย (Production) ซึ่งจะช่วยลดต้นทุนในการแก้ไขปัญหา
    
    \item \textbf{Continuous Monitoring}: การตรวจสอบและเฝ้าระวังความปลอดภัยอย่างต่อเนื่องตลอดวงจรชีวิตของแอปพลิเคชัน
    
    \item \textbf{Collaboration}: การสร้างวัฒนธรรมความร่วมมือระหว่างทีมพัฒนา ทีมปฏิบัติการ และทีมความปลอดภัย
\end{enumerate}

หลักการ Shift-Left Security มีความสำคัญอย่างยิ่ง เนื่องจากการตรวจพบช่องโหว่ในระยะแรก ๆ ของการพัฒนาจะช่วยลดต้นทุนในการแก้ไขได้มากกว่าการค้นพบในระยะหลัง ตามการศึกษาพบว่าต้นทุนในการแก้ไขช่องโหว่ในระหว่างการพัฒนาต่ำกว่าการแก้ไขในระบบที่ทำงานจริงถึง 30 เท่า

\section{วิธีการตรวจสอบความปลอดภัยแอปพลิเคชัน}

การตรวจสอบความปลอดภัยของแอปพลิเคชันมีหลายวิธีการ โดยแต่ละวิธีมีจุดแข็งและจุดอ่อนที่แตกต่างกัน ตามคำแนะนำของ NIST SP 800-115~\cite{nist800-115} การตรวจสอบความปลอดภัยที่ครอบคลุมควรใช้หลายวิธีการร่วมกัน

\subsection{Static Application Security Testing (SAST)}

SAST เป็นวิธีการตรวจสอบความปลอดภัยโดยวิเคราะห์ซอร์สโค้ดหรือ Compiled Code โดยไม่ต้องรันแอปพลิเคชัน~\cite{sast2018} ข้อดีของ SAST คือ:

\begin{itemize}
    \item สามารถตรวจจับปัญหาได้ในระยะแรกของการพัฒนา
    \item วิเคราะห์โค้ดได้ครอบคลุมทุก Path ไม่พลาดส่วนที่ไม่ค่อยถูกเรียกใช้
    \item ระบุตำแหน่งที่แน่นอนของช่องโหว่ในโค้ด ทำให้แก้ไขได้ง่าย
    \item ไม่ต้องการสภาพแวดล้อมที่รันได้จริง
\end{itemize}

ข้อจำกัดของ SAST:
\begin{itemize}
    \item อาจมี False Positive สูง ต้องใช้เวลาในการกรองผลลัพธ์
    \item ไม่สามารถตรวจจับปัญหาที่เกิดจากการตั้งค่า Runtime หรือ Infrastructure
    \item ต้องการความเข้าใจเกี่ยวกับภาษาและ Framework ที่ใช้
\end{itemize}

\subsection{Dynamic Application Security Testing (DAST)}

DAST เป็นวิธีการตรวจสอบโดยการทดสอบแอปพลิเคชันที่กำลังทำงานจริง (Black-box Testing) โดยไม่ต้องเข้าถึงซอร์สโค้ด ข้อดีของ DAST:

\begin{itemize}
    \item ตรวจจับปัญหาที่เกิดขึ้นในสภาพแวดล้อมจริง
    \item มี False Positive ต่ำกว่า SAST
    \item ไม่ต้องการความรู้เกี่ยวกับภาษาหรือเทคโนโลยีที่ใช้พัฒนา
\end{itemize}

ข้อจำกัดของ DAST:
\begin{itemize}
    \item ตรวจจับปัญหาได้ช้ากว่า SAST เนื่องจากทำในระยะหลัง
    \item อาจพลาด Code Path ที่ไม่ถูกเรียกใช้ในระหว่างการทดสอบ
    \item ไม่สามารถระบุตำแหน่งที่แน่นอนในโค้ด
\end{itemize}

\subsection{Software Composition Analysis (SCA)}

SCA เป็นเครื่องมือสำหรับตรวจสอบช่องโหว่ใน Third-party Libraries และ Dependencies ที่แอปพลิเคชันใช้งาน โดยเปรียบเทียบกับฐานข้อมูลช่องโหว่ที่ทราบ เช่น National Vulnerability Database (NVD) ข้อดีของ SCA:

\begin{itemize}
    \item ตรวจจับช่องโหว่ใน Dependencies ที่นักพัฒนาอาจไม่รู้
    \item ให้ข้อมูลเกี่ยวกับ License Compliance
    \item ทำงานได้รวดเร็ว เนื่องจากเปรียบเทียบกับฐานข้อมูลที่มีอยู่
\end{itemize}

ระบบ VisScan ใช้แนวทางผสมผสานระหว่าง SAST (ผ่าน Semgrep) และ SCA (ผ่าน Trivy) เพื่อให้ได้ผลการตรวจสอบที่ครอบคลุมมากที่สุด

\section{มาตรฐานและแนวทางด้านความปลอดภัย}

\subsection{ISO/IEC 27001:2022 - Information Security Management}

ISO/IEC 27001~\cite{iso27001} เป็นมาตรฐานสากลสำหรับระบบการจัดการความมั่นคงปลอดภัยสารสนเทศ (Information Security Management System - ISMS) ที่กำหนดข้อกำหนดในการสร้าง ดำเนินการ บำรุงรักษา และปรับปรุงระบบจัดการความปลอดภัยอย่างต่อเนื่อง

หลักการสำคัญของ ISO/IEC 27001 ที่เกี่ยวข้องกับ VisScan:

\begin{enumerate}
    \item \textbf{Vulnerability Management (A.8.8)}: องค์กรต้องมีกระบวนการระบุ ประเมิน และจัดการกับช่องโหว่อย่างเป็นระบบ VisScan ตอบสนองข้อกำหนดนี้โดยการสแกนและรายงานช่องโหว่อย่างต่อเนื่อง
    
    \item \textbf{Secure Development Lifecycle (A.8.25)}: การพัฒนาซอฟต์แวร์ต้องคำนึงถึงความปลอดภัยในทุกขั้นตอน VisScan สนับสนุนโดยการตรวจสอบโค้ดและคอนเทนเนอร์ตั้งแต่ระยะพัฒนา
    
    \item \textbf{Information Security in Project Management (A.5.8)}: โครงการพัฒนาซอฟต์แวร์ต้องรวมความปลอดภัยเป็นส่วนหนึ่งของกระบวนการ
\end{enumerate}

\subsection{NIST SP 800-190 - Application Container Security Guide}

NIST SP 800-190~\cite{nist800-190} เป็นคู่มือด้านความปลอดภัยสำหรับ Application Container ที่จัดทำโดย National Institute of Standards and Technology ของสหรัฐอเมริกา เอกสารนี้ระบุความเสี่ยงและแนวทางการรักษาความปลอดภัยสำหรับเทคโนโลยีคอนเทนเนอร์

ความเสี่ยงหลักที่ NIST SP 800-190 ระบุ:

\begin{enumerate}
    \item \textbf{Image Vulnerabilities}: ช่องโหว่ในซอฟต์แวร์ที่รวมอยู่ใน Container Image
    \item \textbf{Image Configuration Defects}: การตั้งค่าที่ไม่ปลอดภัยในการสร้าง Image
    \item \textbf{Embedded Malware}: มัลแวร์ที่ถูกฝังไว้ใน Image
    \item \textbf{Embedded Clear Text Secrets}: ข้อมูลลับที่ถูกฝังไว้ใน Image แบบ Plain Text
    \item \textbf{Untrusted Images}: การใช้ Image จากแหล่งที่ไม่น่าเชื่อถือ
\end{enumerate}

แนวทางที่แนะนำ:
\begin{itemize}
    \item สแกนหาช่องโหว่ใน Image ก่อน Deploy
    \item ใช้ Image จาก Trusted Registry เท่านั้น
    \item อัปเดต Image และ Base Image อย่างสม่ำเสมอ
    \item ไม่ฝังข้อมูลลับไว้ใน Image โดยตรง
    \item ใช้ Principle of Least Privilege ในการตั้งค่า Container
\end{itemize}

VisScan ตอบสนองต่อแนวทางเหล่านี้โดยการใช้ Trivy สแกน Container Image และ Gitleaks ตรวจหาข้อมูลลับที่อาจถูกฝังไว้

\subsection{OWASP Top 10}

OWASP Top 10~\cite{owasp2021} เป็นรายการช่องโหว่ด้านความปลอดภัยที่พบบ่อยและร้ายแรงที่สุดในแอปพลิเคชันเว็บ จัดทำโดย Open Web Application Security Project รายการล่าสุด (2021) ประกอบด้วย:

\begin{enumerate}
    \item \textbf{A01:2021 - Broken Access Control}: ปัญหาการควบคุมการเข้าถึง
    \item \textbf{A02:2021 - Cryptographic Failures}: ความล้มเหลวในการเข้ารหัส
    \item \textbf{A03:2021 - Injection}: ช่องโหว่ Injection เช่น SQL Injection
    \item \textbf{A04:2021 - Insecure Design}: การออกแบบที่ไม่ปลอดภัย
    \item \textbf{A05:2021 - Security Misconfiguration}: การตั้งค่าที่ไม่ปลอดภัย
    \item \textbf{A06:2021 - Vulnerable and Outdated Components}: การใช้ส่วนประกอบที่มีช่องโหว่
    \item \textbf{A07:2021 - Identification and Authentication Failures}: ปัญหาการยืนยันตัวตน
    \item \textbf{A08:2021 - Software and Data Integrity Failures}: ปัญหาความสมบูรณ์ของข้อมูล
    \item \textbf{A09:2021 - Security Logging and Monitoring Failures}: ปัญหาการบันทึกและเฝ้าระวัง
    \item \textbf{A10:2021 - Server-Side Request Forgery (SSRF)}: ช่องโหว่ SSRF
\end{enumerate}

Semgrep ที่ใช้ใน VisScan สามารถตรวจจับช่องโหว่หลายประเภทที่อยู่ใน OWASP Top 10 โดยเฉพาะ Injection, Security Misconfiguration และ Vulnerable Components

\section{เครื่องมือตรวจสอบความปลอดภัย}

\subsection{Trivy}

Trivy~\cite{trivy} เป็น Open-source Vulnerability Scanner ที่พัฒนาโดย Aqua Security มีความสามารถหลักดังนี้:

\begin{itemize}
    \item \textbf{Container Image Scanning}: สแกนช่องโหว่ใน Container Image รองรับหลายรูปแบบ เช่น Docker, OCI
    \item \textbf{Filesystem Scanning}: สแกนระบบไฟล์และ Dependencies
    \item \textbf{Git Repository Scanning}: สแกนช่องโหว่โดยตรงจาก Git Repository
    \item \textbf{Kubernetes Scanning}: ตรวจสอบการตั้งค่า Kubernetes
    \item \textbf{Multi-language Support}: รองรับหลายภาษา เช่น Go, Python, JavaScript, Java, Ruby
\end{itemize}

Trivy ใช้ฐานข้อมูลช่องโหว่จากหลายแหล่ง:
\begin{itemize}
    \item National Vulnerability Database (NVD)
    \item Red Hat Security Data
    \item Debian Security Bug Tracker
    \item Alpine Linux Security Database
    \item และอื่น ๆ
\end{itemize}

ข้อดีของ Trivy:
\begin{itemize}
    \item รวดเร็ว สแกน Image ได้ภายในไม่กี่วินาที
    \item ฐานข้อมูลช่องโหว่ครอบคลุมและอัปเดตอัตโนมัติ
    \item รองรับหลาย Output Format (JSON, SARIF, Table)
    \item ใช้งานง่าย ไม่ต้องติดตั้งฐานข้อมูลเพิ่มเติม
    \item ตรวจจับ Misconfiguration ในไฟล์ IaC (Terraform, Kubernetes YAML)
\end{itemize}

\subsection{Gitleaks}

Gitleaks~\cite{gitleaks} เป็นเครื่องมือ SAST สำหรับการตรวจจับข้อมูลลับที่รั่วไหล (Secret Detection) ใน Git Repository โดยสแกนหา:

\begin{itemize}
    \item API Keys และ Access Tokens
    \item Database Credentials (Username/Password)
    \item Private Keys (SSH, TLS/SSL)
    \item OAuth Tokens
    \item AWS Credentials
    \item Google Cloud Keys
    \item และข้อมูลลับอื่น ๆ
\end{itemize}

คุณสมบัติสำคัญ:
\begin{itemize}
    \item \textbf{Pre-commit Scanning}: สามารถใช้เป็น Git Hook ตรวจสอบก่อน Commit
    \item \textbf{Historical Scanning}: สแกน Git History ทั้งหมดหาข้อมูลลับที่ผ่านมา
    \item \textbf{Custom Rules}: รองรับการสร้าง Custom Rules สำหรับตรวจจับรูปแบบเฉพาะ
    \item \textbf{Fast Performance}: ใช้ Algorithm ที่มีประสิทธิภาพสูง
\end{itemize}

Gitleaks ใช้ Regular Expression และ Entropy Analysis เพื่อตรวจจับข้อมูลลับ โดยมี Ruleset ที่ครอบคลุมสำหรับบริการและแพลตฟอร์มยอดนิยม

\subsection{Semgrep}

Semgrep~\cite{semgrep} เป็น Static Analysis Engine ที่พัฒนาโดย Semgrep Inc. (r2c) เป็นเครื่องมือที่ใช้ Pattern Matching แบบ Lightweight เพื่อหาจุดที่มีปัญหาในโค้ด

คุณสมบัติหลัก:
\begin{itemize}
    \item \textbf{Multi-language Support}: รองรับมากกว่า 30 ภาษา รวมถึง Python, JavaScript, TypeScript, Java, Go, C, C++, Ruby
    \item \textbf{Easy Rule Writing}: ใช้ Syntax ที่คล้ายกับภาษาโปรแกรมจริง ทำให้เขียน Rule ได้ง่าย
    \item \textbf{Fast Scanning}: เร็วกว่า Traditional SAST Tools หลายเท่า
    \item \textbf{Low False Positives}: มีอัตรา False Positive ต่ำ
    \item \textbf{Security Rules}: มี Ruleset สำเร็จรูปจาก OWASP Top 10 และ CWE
\end{itemize}

Semgrep Registry ให้บริการ Rules มากกว่า 2,000 Rules ที่ครอบคลุม:
\begin{itemize}
    \item Security Vulnerabilities (OWASP Top 10)
    \item Code Quality Issues
    \item Performance Problems
    \item Common Anti-patterns
\end{itemize}

ข้อดีเมื่อเทียบกับเครื่องมืออื่น:
\begin{itemize}
    \item เร็วกว่า Traditional SAST เช่น Fortify, Checkmarx
    \item ตั้งค่าง่ายกว่า Complex SAST Tools
    \item รองรับ CI/CD Integration ได้ดี
    \item Community-driven Rules
\end{itemize}

\section{เทคโนโลยี Container และ Orchestration}

\subsection{Docker}

Docker~\cite{docker2014} เป็นแพลตฟอร์ม Container ที่ใช้ในการสร้าง Deploy และรันแอปพลิเคชันในสภาพแวดล้อมที่แยกออกมา (Isolated Environment) ความสำคัญของ Docker ต่อ VisScan:

\begin{itemize}
    \item ใช้รันเครื่องมือสแกนแบบ Isolated ป้องกันผลกระทบต่อระบบหลัก
    \item รองรับการสแกน Container Image เพื่อหาช่องโหว่
    \item ใช้ในการ Deploy ระบบ VisScan เอง
\end{itemize}

\subsection{Kubernetes}

Kubernetes~\cite{kubernetes2016} เป็น Container Orchestration Platform ที่ใช้จัดการ Container ในสภาพแวดล้อม Production เกี่ยวข้องกับ VisScan ในด้าน:

\begin{itemize}
    \item การ Deploy VisScan บน Kubernetes Cluster
    \item การสแกนความปลอดภัยของ Kubernetes Configuration
    \item การบูรณาการกับ Admission Controller เพื่อตรวจสอบก่อน Deploy
\end{itemize}

\section{สรุป}

ทฤษฎีและเทคโนโลยีที่กล่าวมาในบทนี้เป็นพื้นฐานสำคัญในการพัฒนาระบบ VisScan โดยเฉพาะหลักการ DevSecOps, มาตรฐานสากล (ISO/IEC 27001, NIST SP 800-190), และเครื่องมือสแกนความปลอดภัย (Trivy, Gitleaks, Semgrep) การนำความรู้เหล่านี้มาประยุกต์ใช้จะทำให้ระบบมีความสมบูรณ์และตอบสนองความต้องการด้านความปลอดภัยขององค์กรได้อย่างมีประสิทธิภาพ





\section{\ifenglish%
\ifcpe CPE \else ISNE \fi knowledge used, applied, or integrated in this project
\else%
ความรู้ตามหลักสูตรซึ่งถูกนำมาใช้หรือบูรณาการในโครงงาน
\fi
}

โครงงาน VisScan ได้นำความรู้จากหลักสูตรวิศวกรรมซอฟต์แวร์มาใช้ในการวางแผนและพัฒนาระบบ เช่น การจัดการวงจรชีวิตซอฟต์แวร์ (SDLC) การใช้แนวคิด Agile ในการแบ่งงานและปรับปรุงอย่างต่อเนื่อง รวมถึงการออกแบบฐานข้อมูล PostgreSQL โดยอาศัยหลักการของวิชา Database Systems เพื่อให้สามารถจัดเก็บและบริหารข้อมูลได้อย่างมีประสิทธิภาพ นอกจากนี้ยังประยุกต์ใช้ความรู้ด้านเครือข่ายคอมพิวเตอร์ เช่น การออกแบบ RESTful API และการสื่อสารผ่านโปรโตคอล HTTP เพื่อเชื่อมต่อระบบ Frontend และ Backend อย่างปลอดภัยและมีประสิทธิภาพ


\section{\ifenglish%
Extracurricular knowledge used, applied, or integrated in this project
\else%
ความรู้นอกหลักสูตรซึ่งถูกนำมาใช้หรือบูรณาการในโครงงาน
\fi
}

ในการพัฒนาโครงงาน VisScan นักศึกษาได้ศึกษาค้นคว้าความรู้เพิ่มเติมนอกหลักสูตรเกี่ยวกับแนวคิด DevSecOps การรักษาความปลอดภัยของคอนเทนเนอร์ (Container Security) โดยเฉพาะการใช้งาน Docker และ Kubernetes รวมถึงการบูรณาการเครื่องมือด้านความปลอดภัย เช่น Trivy, Gitleaks และ Semgrep เข้ากับเฟรมเวิร์ก Next.js เพื่อให้ระบบสามารถตรวจสอบและรายงานช่องโหว่ได้อย่างมีประสิทธิภาพและสอดคล้องกับมาตรฐานอุตสาหกรรม
